% Chapter Template

\chapter{Conclusion}
\label{Chapter6}

I have implemented three separate argument mining techniques, applicable to both structural and statistical approach. For the \textbf{structural approach}, there was created a dictionary of lexical cues that usually characterize argumentative structures based on (\cite{knott1994using}). The algorithm created was tested in two IBM corpora containing only argumentative sentences, the accuracy seems to be around \textbf{15 to 17 \%}. As regards the statistical approach, a corpora has to be curated in order to include both argumentative and non-argumentative instances. That is why Wikipedia articles were scrapped assuming that there are no arguments included, and the sentences gathered were cross-checked by the algorithm created for the structural approach. Every sentence that did not have argumentative cues, based on the algorithm, was added to the corpora as non-argumentative, while the others considered as ambiguous and were checked by two annotators so as to remove misleading indicators of dictionary. Afterwards, two algorithms were built by using as training data a part of this corpora. \textbf{Random forest classification algorithm} was approached by two different methods of training. The first technique was by using the tokenized sentences which leads to an accuracy of \textbf{56.83\%}, while the second was by using features like counter of words, uppercase and punctuation characters that have an accuracy of \textbf{79.42\%}. Finally, the other algorithm of statistical approach that was built is the \textbf{LSTM-RNN} that concludes to an accuracy of \textbf{85.44\%}. 

As the results of both structural and statistical algorithms reveal, linguistic rules are not capable of successfully identifying arguments. On the other hand, learning algorithms are a better fit to argument mining, with the most accurate one to be the LSTM-RNN algorithm. One of the main goals as regarding future work is to apply the algorithms built in a fully annotated data-set found in GitHub's issues.