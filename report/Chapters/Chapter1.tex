\chapter{Introduction}

\label{Chapter1}

\section{Definition} 

Argument mining is a relatively new research field in natural language processing. The aim of this research is the auto detection and identification of argumentative structures expressed in text. In order to perform extraction and evaluation of arguments, computer science and artificial intelligence is used. \par

An argument is a group of premises conducted to support a claim (\cite{Palau2009}). When it comes to real world, arguments are hardly identified even by experts (\cite{Lippi2015}). The ambiguity of natural language, the implicit content, the different ways of expressing and the complex structure of arguments are the main reasons why argument mining is a challenging research field. Labeled corpora are scarce which is a fact that slows down field's potential growth (\cite{Lippi2015}). \par

The purpose of argument mining is to understand what kind of views have been expressed in the examined text and why they are held. Argument mining has derived from opinion mining and sentiment analysis  research area, in which the only goal is to understand the opinions about a certain topic (\cite{Lawrence2015}). \par

\section{Research Goal}
 My research goal is to identify argumentative statements by using two different approaches; the structural approach which is based on hand coded rules and the statistical approach, which based on supervised and deep learning algorithms. \par
 
 The structural approach uses lexical cues that have been identified by linguists as signs of argumentative speech. As an example, words such as "because", "therefore", "in order to" are common cues of arguments. However, these argumentative patterns are rarely used in practice,	since human discourse involves a lot of information which is being implied rather than being explicitly stated. \par
 
 On the other hand, the statistical approach relies on examples of pieces of text that have been manually labeled as argumentative or non-argumentative. These are used for training models in order to automatically identify arguments in free text without the use of predefined lexical cues and rules. The challenging part is the construction of a manually annotated data-set, given the fact that a large amount of data are required for training such models. \par
 
 The fundamental research questions that will be addressed in this assignment are the following:
 \begin{itemize}
   \item To what extent are the lexical rules drafted by a structural approach capable of successfully identifying arguments in existing resources of labeled data?
   \item Do the statistical approaches outperform these results?
 \end{itemize}  


\section{Assignment's Structure}

This paper of research is organized into 6 chapters. Chapter \ref{Chapter2} presents the state of the art in argument mining, and introduces the two different approaches; the structural and the machine learning approach. Chapters \ref{Chapter3} and \ref{Chapter4} describe in detail the methods and results of both approaches implemented in the scope of this study. Chapter \ref{Chapter5} contains the corpora created for the supervised algorithm, while chapter \ref{Chapter6} concludes with a look to future work. \par
