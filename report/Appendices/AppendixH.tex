% Appendix 8

\chapter{Find Argumentative Sentences}

\label{Appendix8}

\begin{lstlisting}[language=iPython]
import json
import spacy
from __future__ import division
from configParser import choose_function, os, py23_str, re

"""
Description: by using a dictionary that includes words often used 
in arguments, this file identifies if given sentences are 
arguments or not. Part of speech tagging from spaCy is used 
for this purpose as well.
"""

FILE_PATH = os.path.abspath(os.path.dirname(__file__))

tp = 0
tn = 0
fp = 0
fn = 0


def spaCy(sentence):
	"""
	By using spaCy, this function gets a sentence and returns every
	word's part of speech
	
	:param sentence: input to be tokenized
	:return: tokenized sentence
	"""
	
	nlp = spacy.load('en')
	doc = nlp(py23_str(sentence))
	
	return doc


def pos_tagged(doc, word):
	"""
	This function gets a tagged sentence from spaCy and a specific 
	word and return its part of speech and its dependency
	
	:param doc: pos tagged sentence from spacy function
	:param word: a word that we are interested to learn its part of 
		speech
	:return: word, its part of speech(pos) and its dependence in the
		given sentence or None
	"""
	
	word = word.lower()
	
	for token in doc:
		if token.text.lower() == word:
			return [token.text, token.pos_.lower(), token.dep_.lower()]
	
	return 'None'


def check_regex(doc_regex, regex):
	"""
	By using spaCy's function called match, this function is
	checking if a specific regular expression is represented
	by a given sentence
	
	:param doc_regex: pos tagged sentence from spacy function
	:param regex: a regular expression
	:return: the part of the sentence that is indicated in the given
		regex otherwise None
	"""
	
	regex = re.compile(r''+regex)
	
	for match in re.finditer(regex, doc_regex.text.lower()):
		start, end = match.span()  # get matched indices
		word_found = doc_regex.char_span(start, end)  # create Span from indices
		
		return word_found
	
	return None


def check_dictionary(doc, dictionary):
	"""
	This function checks if any of the words in the sentence exists
	 in the dictionary given. If it does, then it is checked if 
	 this word's part of speech match with its value given in 
	 dictionary. If they match, then the word is added in a 
	 list named keyword_found.
	
	:param doc: pos tagged sentence from spacy function
	:param dictionary: dictionary that has as keywords words 
		and as value their part of speech
	:return: True if the list keyword_found is not empty or False 
			 if it is empty
	"""
	
	keyword_found = []
	
	for key, value in dictionary.items():
		if check_regex(doc, key) is not None:
			if len(value) == 1:
				for key2 in value:
					if checz_regex(doc, key + ' ' + key2) is not None 
					 and check_regex(doc, key2) is not None and 
					 pos_tagged(doc, check_regex(doc, key2).text) is not 'None':
						if value[key2] == pos_tagged(doc, check_regex(doc, key2).text)[1] or \
						 value[key2] == pos_tagged(doc, check_regex(doc, key2).text)[2]:
							keyword_found.append(key + " + " + key2)
			elif value == 'none':
				keyword_found.append(key)
			elif len(value) != 1 and check_regex(doc, value) 
			 is not None:
				keyword_found.append(key)
			elif pos_tagged(doc, key)[1] == value or \
			 pos_tagged(doc, key)[2] == value:
				keyword_found.append(key)
	
	if len(keyword_found) != 0:
		return 'True'
	else:
		return 'False'


def check_validity(real_value, given_value):
	"""
	This function checks if two given values match
	
	:param real_value:  real value is the result of check_dictionary
		function
	:param given_value: given value is the value given by analysts 
		into dataset
	:return: Correct results if they match or Wrong results if they
		do not match
	"""
	global tn, tp, fn, fp
	
	if real_value in given_value:
		if real_value == 'True':
			tp = tp + 1
		else:
			tn = tn + 1
		return "Correct results"
	else:
		if real_value == 'False':
			fp = fp + 1
		else:
			fn = fn + 1
		return "Wrong results"


def precision():
	if (tp + fp) != 0:
		return tp/(tp + fp)
	else:
		return "Integer division by zero"


def recall():
	if (tp + fn) != 0:
		return tp/(tp + fn)
	else:
		return "Integer division by zero"


def f1_score(precision, recall):
	if (precision + recall) != 0:
		return 2 * (precision * recall) / (precision + recall)
	else:
		return "Integer division by zero"


if __name__ == '__main__':

	with open('../dict/dictionary.json', 'r') as dict:
		dictionary = json.load(dict)
	
	sentences = choose_function("found_fp.csv")
	labeled_data = False
	
	if sentences != 'No dataset found':
        for sentence in sentences:

			if len(sentence) == 2:
				print('"' + str(sentence[0]).strip('b') + '",' + 'True') # for csv
				# print ( str ( sentence[0] ).strip ( 'b' ) + ',' + 'True' )
				labeled_data = True
				
			else:
				print('"' + str(sentence[0]).strip('b') + '",' + 
				 check_dictionary(spaCy(sentence[0]), dictionary)) # for csv
	
	else:
		print(sentences)
\end{lstlisting}
