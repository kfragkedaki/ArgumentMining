% Appendix 9

\chapter{Statistical Approach: Random Forest algorithm}

\label{Appendix9}

\begin{lstlisting}[language=iPython]
import pandas as pd
import seaborn as sn
from sklearn.preprocessing import OneHotEncoder
from sklearn.model_selection import train_test_split
from sklearn.ensemble import RandomForestClassifier
from sklearn.metrics import confusion_matrix
import matplotlib.pyplot as plt
import string
import csv
import os
import numpy as np


"""
Description: implementing Random Forest classifier model
to an annotated corpora that contains 
both argumentative and none sentences
"""

FILE_PATH = os.path.abspath(os.path.dirname(__file__))  # path of this file


def load_dataset(path):
	"""
	Loading dataset of a given path, and creating features based on the sentences given.
	The first feature is a counter of words included in each sentence,
	the second feature is a counter of uppercase characters, while
	the third feature is a counter of special characters (punction)
	:param path: full path of dataset
	:return: a list of sentences and their labels, and a set of features
	"""
	
	# load & prepare data
	with open(path, 'r') as file:
		dataset = csv.reader(file, delimiter=",")
		df = pd.DataFrame(dataset)
	
	del df[2]  # column full of non	labels = df[1]
	df[1] = [len(sentences.split()) for sentences in df[0]]  # Word Count'
	df[2] = [sum(char.isupper() for char in sentence) \
			for sentence in df[0]]  # 'Uppercase Char Count'
	
	df[3] = [sum(char in string.punctuation for char in sentence) \
			for sentence in df[0]]  # 'Special Char Count'
	df[4] = pd.factorize(labels)[0]  # switch False to 0 and True to 1
	
	return df


def tokenize_sentences(df):
	"""
	Tokenizing raw sentences by using One-hot encoding
	into a format that a computer can understand
	
	:param df: the dataframe that includes 4 columns
				[sentences, word Counter, uppercase counter,
				special char counter, label]
	:return: tokenized sentences, labels
	"""
	vectorizer = OneHotEncoder(handle_unknown='ignore')
	vectorizer.fit([[line.strip()] for line in df[0]])
	sentences = vectorizer.transform([[line.strip()] for line in df[0]]).toarray()
	
	# sentences = vectorizer.fit_transform(df[0])
	labels = df[4]
	return sentences, labels


def define_model(x_train, y_train):
	"""
	This function defines and trains the Random Forest model
	
	:param x_train: the training sentences
	:param y_train: the sentences' labels
	:return: trained model
	"""
	model = RandomForestClassifier()
	model.fit(x_train, y_train)
	return model


def precision(tp, fp):
	if (tp + fp) != 0:
		return tp/(tp + fp)
	else:
		return "Integer division by zero"


def recall(tp, fn):
	if (tp + fn) != 0:
		return tp/(tp + fn)
	else:
		return "Integer division by zero"


def f1_score(precision, recall):
	if type(recall) != str and type(precision) != str and (precision + recall) != 0:
		return 2 * (precision * recall) / (precision + recall)
	else:
		return "Integer division by zero"


def testing_model(model, x_test, y_test):
	"""
	This function evaluates the model on the test set
	
	:param model: trained model
	:param x_test: the testing set of sentences
	:param y_test: the sentences' labels
	:return: a matrix of the predicted and real values
	"""
	score = model.score(x_test, y_test)
	print("Accuracy: %.2f%%" % (score * 100))
	
	y_predicted = model.predict(x_test)
	cm = confusion_matrix(y_test, y_predicted)
	
	fp = cm[0][1]
	fn = cm[1][0]
	tp = cm[1][1]
	
	prec = precision(tp, fp)
	rec = recall(tp, fn)
	
	print("Precision: %.2f%%" % (prec * 100))
	print("Recall: %.2f%%" % (rec * 100))
	print("F1 Score: %.2f%%" % (f1_score(prec, rec) * 100) )
	
	print(cm)
	return cm


def plot_results(cm):
	"""
	This function is showing a heatmap plot based on
	the values of the confusion matrix that contains
	the real an predicted values.
	:param cm: matrix of both real and predicted values
	"""
	plt.figure(figsize=(10, 7))
	sn.heatmap(cm, annot=True)
	plt.xlabel('Prediction')
	plt.ylabel('Truth')
	plt.show()


if __name__ == '__main__':
	df = load_dataset('../Results/dataset.csv')
	sentences, labels = tokenize_sentences(df)
	
	#  using features instead of the sentences
	#
	features = np.asarray(df[df.columns[1:4]].values)
	# split dataset into test and train data
	x_train, x_test, y_train, y_test = \
		train_test_split(features, labels, test_size=0.33)
	
	model = define_model(x_train, y_train)
	cm = testing_model(model, x_test, y_test)
	plot_results(cm)
	
	#  using sentences without their features
	#
	# split dataset into test and train data
	x_train, x_test, y_train, y_test = train_test_split(sentences, labels, test_size=0.33)
	
	model = define_model(x_train, y_train)
	cm = testing_model(model, x_test, y_test)
	plot_results(cm)

\end{lstlisting}
